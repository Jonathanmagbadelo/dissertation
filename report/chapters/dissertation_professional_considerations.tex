%-----------------------------------------------------
% Chapter: Professional Considerations
%-----
\chapter{Professional Considerations}
\label{chap:prof_con}
Throughout the development of this project both professional and ethical considerations were taken into account, including those highlighted in the British Computing Societies (BCS) Code of Conduct \footnote{https://www.bcs.org/category/6030} and Code of Good Practice. This chapter outlines the relevant areas in the specified documents which have been adhered to during the project.

\section{Code of Conduct}
\subsubsection{Professional Competence and Integrity}
The completion of this project was a large undertaking due to the implementation and integration of a novel machine learning method within a prototype software application. Though the project is beyond the scope of a typical final year project, all work carried out have roots to modules taken in the University of Sussex Computer Science and Artificial Intelligence course, specifically the Neural Networks, Advance Natural Language Engineering and Software Engineering modules. In accordance with section 2.C of the BCS Code of Conduct, background research continually occurred throughout development to maintain a competent standard of professional knowledge.

\subsubsection{Duty of Relevant Authority}
In agreement with section 3.A and 3.B of the BCS Code of Conduct, all scenarios which may cause a conflict of interest between the project and the University of Sussex have been avoided.

\subsubsection{Duty to Professionalism}
In accordance with section 4.A and 4.C of the BCS Code of Conduct, the manner in which this project was conducted was one which maintained the reputation of the BCS. During meetings with other BCS members and professionals such as my project supervisor and work colleagues, appropriate levels of respect and integrity were upheld in accordance with section 4.B of the BCS Code of Conduct.

\section{Good Practices}
The motivation behind this project is one rooted in exploratory research rather than being client driven. Nevertheless, it is important that code produced is well structured and testable to ensure quality assurance. Where possible and in accordance with section 5.2 of the BCS Code of Good Practice, the code produced is well structured and organised to help facilitate further testing and maintainability. 

\noindent
\newline
The same section of the BCS Code of Good Practice refers to the following of programming language guidelines. Both Python and JavaScript were used extensively during the development of the project and where appropriate best practices and coding style/conventions have been adhered to. 

\section{Ethical Considerations}
The success of a machine learning project relies heavily on data availability and quality. Currently, there exists no central repository for downloadable song lyrics due to potential copyright infringement. Under the Research and Private Study section of the Copyright, Designs and Patents Act (1988) \footnote{https://www.gov.uk/government/publications/copyright-acts-and-related-laws}, a public song lyric dataset is used solely for training machine learning models.  

\noindent
\newline
Finally, the project utilises textual data which may have explicit or offensive content within it. After the training of models, which will not be filtered to allow permit artistic freedom, a filtering option will be implemented in order to prevent potential users from seeing unwanted content.