%-----------------------------------------------------
% Chapter: Related Work
%-----------------------------------------------------
\chapter{Related Work}
\label{chap:related_work}
This chapter briefly outlines previous work relating to the tasks of language modelling and text classification in the area of music lyrics.

\section{Lyric Language Modelling}
Generally, hobbyist approaches to modelling the language used in lyrics have utilized Markov processes, which have the special property that the future is independent of the past given the present. Naturally, as only the present or recent past (in the case of higher order Markov chains) is considered, these types of models are unable to capture longer dependencies and complex structures, such as alliteration and complex rhyme schemes; examples of which are given below:

\begin{figure}[ht]
	\begin{center}
		"Tired of \textbf{injustice}
		\par
		Tired of the \textbf{schemes}
		\par
		Your lies are \textbf{disgusting}
		\par
		What does it \textbf{mean}
		\par
		Kicking me \textbf{down}
		\par
		I gotta get \textbf{up}
		\par
		As jacked as it \textbf{sounds}
		\par
		The whole system \textbf{sucks}"
	\end{center}
	\caption[Example of complex rhyme in Michael Jackson - Scream]{Example of complex rhyme in Michael Jackson - Scream. In this example alternate slant rhymes are used following the \textit{ababcdcd} rhyme scheme.}
\end{figure}

\begin{figure}[ht]
	\begin{center}
		"Hear Her Voice
		\par
		Shake My Window
		\par
		\textbf{S}weet \textbf{S}educing \textbf{S}ighs"
	\end{center}
	\caption[Example of alliteration in Michael Jackson - Human Nature]{Example of alliteration in Michael Jackson - Human Nature. In this example alliteration of the letter S is used in the last line.}
\end{figure}



\noindent
\newline
Recent attempts to model creative writing have utilised neural methods, such as \cite{Zhang2014}, who uses RNN's for Chinese poetry generation and \cite{Potash2015}, who uses an LSTM, to \textit{'ghostwrite'}\footnote{write in the style of X without giving credit} rap lyrics. 

\section{Lyric Genre Classification}
Previous attempts at genre classification predominately use audio features as opposed to lyrics, (in part due to lyric availability). Apart from audio features, the genre of a song is also connected to its lyrics as songs of varying genres use language differently. This can be seen in the Hip Hop, where the use of slang, which are twisted existing words or self-made words (\cite{Edwards2009}) is prevalent. Examples of such words can be seen in \autoref{Tab:Slang}
\begin{table}[ht]
	\centering
	\begin{tabular}{ | p{5cm} | p{5cm} |}
		\hline
		\textbf{Slang Word} & \textbf{Meaning}\\ \hline
		grill & teeth\\ \hline
		lit & fun\\ \hline
		bread & money \\ \hline
	\end{tabular}
	\caption{Example slang words found in the dataset used in the project}
	\label{Tab:Slang}
\end{table}
\newline
Similar to the previous section, genre classification has also been tackled using neural methods. Using audio based features, (\cite{Irvin2016}) and (\cite{Pui2018}), utilise LSTM's for classification. In relation to song lyrics \cite{Tsaptsinos2017} utilises lyrics to train a Hierarchical Attention Network (\cite{Yang2016}), which provide state of the art results for document classification due to their ability to encode its knowledge about the composites of a document in a hierarchical way. In the same paper, a baseline LSTM is implemented for comparative reasons. 