%-----------------------------------------------------
% Chapter: Professional Considerations
%-----
\chapter{Professional Considerations}
\label{prof_con}
Throughout the development of this project professional and ethical considerations will be considered: including those highlighted in the British Computing Societies (BCS) Code of Conduct 4 and Code of Good Practice. The following sections outline relevant areas in the specified documents as well as how this project will adhere to them:
\section{Code of Conduct}
\section{Good Practices}
\section{Ethical Considerations}
The success of this project relies heavily on the availability of correctly labelled lyric data, and as stated before, there is no central repository where lyrics, along with the required metadata for the project, are stored. Consequently, data collection techniques such as web scraping may be used
to gather additional data.
Though web scraping is common practice, it is important to be a “good citizen” of the web and as such the following will be adhered to:
• robots.txt will be obeyed
• requests to sites will be done politely and if stated in the robots.txt file any request rate
limit will be upheld
• the user agent string of the request will be made explicit for the webmaster to see
Due to the nature of the project, appropriate user testing/research will be carried out during development. In accordance with section 1.a of the BCS Code of Conduct, any private data will be stored securely and if possible anonymised.
The project utilises textual data which may have explicit content within it. After the training of models, which will not be filtered to allow artistic freedom into the model, a filtering option will be implemented in order to prevent potential users from seeing explicit content.